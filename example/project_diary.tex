\documentclass{article}
\usepackage[utf8]{inputenc}
\usepackage{enumitem}
\usepackage{geometry}
\geometry{a4paper, margin=1in}
\title{Project Diary: Lifetime Cost Calculator Extension}
\author{Project Team}
\date{02.04.2025 -- 09.05.2025}

\begin{document}
\maketitle

\section*{Introduction}
This diary documents the development process of the Lifetime Cost Calculator Chrome extension, which calculates and displays the lifetime costs of products (including energy and maintenance) on e-commerce sites. The diary covers the period from 02.04.2025 to 09.05.2025, detailing the steps, decisions, and tools used.

\section*{Project Diary}
\begin{description}[leftmargin=!,labelwidth=2.5cm]

\item[02.04.2025] 
\textbf{Project Kickoff.} \\ 
Initial project planning, requirements gathering, and setup of the project repository. Defined the main goal: a Chrome extension to estimate product lifetime costs on saturn.de.

\item[05.04.2025] 
\textbf{Initial Extension Skeleton.} \\ 
Created the basic Chrome extension structure: \texttt{manifest.json}, \texttt{content.js}, \texttt{popup.html}, and \texttt{options.html}. Set up permissions and content scripts for saturn.de product pages.

\item[10.04.2025] 
\textbf{Product Data Extraction for Saturn.} \\ 
Developed \texttt{saturn_scraper.js} to extract product name, price, and energy data from saturn.de. Integrated this with the content script.

\item[15.04.2025] 
\textbf{Lifetime Cost Calculation Logic.} \\ 
Implemented \texttt{lifetime_calculator.js} to compute energy and maintenance costs over the product lifespan. Added user preferences for electricity rate, discount rate, and appliance lifespans.

\item[20.04.2025] 
\textbf{Popup UI and Options Page.} \\ 
Designed the popup interface (\texttt{popup.html}, \texttt{popup.js}) to display results. Built the options page for user settings and Gemini API key input.

\item[25.04.2025] 
\textbf{Gemini API Integration.} \\ 
Integrated Google Gemini API for fallback product data extraction using page text. Added debug logging for API requests and responses.

\item[28.04.2025] 
\textbf{Testing and Debugging.} \\ 
Tested extension on saturn.de. Fixed issues with product detection, message passing, and error handling in \texttt{options.js} and \texttt{content.js}.

\item[01.05.2025] 
\textbf{Support for digitec.ch.} \\ 
Extended \texttt{manifest.json} to include digitec.ch. Created \texttt{digitec_scraper.js} for product extraction on digitec.ch. Updated \texttt{content.js} to detect digitec.ch product pages and use the new scraper.

\item[03.05.2025] 
\textbf{Improved Product Page Detection.} \\ 
Enhanced logic in \texttt{content.js} to reliably detect product pages on both saturn.de and digitec.ch, including URL and DOM checks.

\item[06.05.2025] 
\textbf{Error Handling and Debugging.} \\ 
Added more robust error handling and debug output in \texttt{content.js} and \texttt{options.js}. Addressed issues with undefined variables and message passing.

\item[09.05.2025] 
\textbf{Final Testing and Documentation.} \\ 
Tested the extension on both supported sites. Created this project diary in LaTeX for documentation and reproducibility. Prepared the project for Overleaf compilation.

\end{description}

\section*{Tools Used}
\begin{itemize}
  \item \textbf{Visual Studio Code} -- Main development environment
  \item \textbf{Chrome Extensions API} -- For browser integration
  \item \textbf{Google Gemini API} -- For AI-based product data extraction
  \item \textbf{JavaScript, HTML, CSS} -- Core technologies
  \item \textbf{Overleaf} -- For collaborative LaTeX documentation
  \item \textbf{Git} -- Version control
\end{itemize}

\end{document}
